\documentclass[25pt]{tikzposter} % See Section 3
%25pt,margin=0mm,innermargin=15mm
\usepackage{amsmath,graphicx}
\newcommand{\hsc}[1]{{\LARGE\MakeUppercase{#1}}}
\newcommand{\hsct}[1]{{\Large\MakeUppercase{#1}}}

%\makeatletter
%\input{theguy28pt.clo}
%\makeatother

\title{\bf Digital Predistortion with Low-Precision ADCs} 
\institute{Rice University} % See Section 4.1
%\titlegraphic{\includegraphics[width=5in]{Rice_logo_hi_res.png}}
\author{Chance Tarver and Joseph R. Cavallaro} 
\usetheme{Basic} % See Section 5

%% THEME STUFF %%
\geometry{paperwidth=42in,paperheight=42in}
\definecolor{riceblue}{RGB}{0,36,106}
\definecolor{ricegray}{RGB}{94,96,98}
%\titlegraphicheight{1mm}
\tikzposterlatexaffectionproofoff  %Turn off watermark thing in corner

%Put logo at the side
\definetitlestyle{sampletitle}{
width=40in, roundedcorners=20, linewidth=2pt, innersep=5pt,
titletotopverticalspace=15mm, titletoblockverticalspace=18mm,
titlegraphictotitledistance = 0pt
}{
\begin{scope}[line width=\titlelinewidth, rounded corners=\titleroundedcorners]
\draw[color=blocktitlebgcolor, fill=titlebgcolor]
(\titleposleft,\titleposbottom) rectangle (\titleposright,\titlepostop);
\end{scope}
}

 \usetitlestyle{sampletitle}
%Recalculate the size of the textwidth
\makeatletter
\setlength{\TP@visibletextwidth}{\textwidth-2\TP@innermargin}
\setlength{\TP@visibletextheight}{\textheight-2\TP@innermargin}
\makeatother

%Set the colors
\colorlet{backgroundcolor}{riceblue}
\colorlet{framecolor}{black}
\colorlet{blocktitlebgcolor}{ricegray} %<---- change color
\colorlet{titlebgcolor}{white}
\colorlet{titlefgcolor}{black}  %Title font color
\colorlet{notebgcolor}{ricegray}
\colorlet{notefgcolor}{white}
\colorlet{noteframecolor}{black}

\makeatletter
\usepackage{helvet}
\renewcommand*{\familydefault}{\sfdefault}% Let's have a sans serif font - See more at: http://latex-cookbook.net/cookbook/examples/poster/#sthash.jDLWH62B.dpuf
\renewcommand\TP@maketitle{%
  \centering
   \begin{minipage}[b]{1\linewidth}
        \centering
        \vspace*{1em}
        \color{black} %color of letters
        {\bfseries \Huge \sc \@title \par}
        \vspace*{1em}
        {\huge \@author \par}
        \vspace*{1em}
        {\LARGE \@institute}
    \end{minipage}%
     \tikz[overlay]\node[scale=1,anchor=east,xshift=16in,yshift=1.5in,inner sep=0pt] {%
       \includegraphics[width=6in]{ECE}
          }; 
      \tikz[overlay]\node[scale=1,anchor=east,xshift=4.4in,yshift=1.6in,inner sep=0pt] {%
       \includegraphics[width=6in]{Rice_logo_hi_res.png}       
   };   
}
\makeatother
\makeatletter
\input{theguy30pt.clo}
\makeatother

\begin{document}


\maketitle % See Section 4.1

\begin{columns} % See Section 4.4
	\centering
	\column{0.33} % See Section 4.4
	\block{Motivation}
	{    \Large
		\begin{itemize}
			\item {\bf Spectrum Scarcity $\rightarrow$ Frequency Agile Standards}
			      \begin{itemize}
				      \item Non-contiguous Transmission
				      \item Carrier Aggregation (CA) in LTE-Advanced
				      \item Cognitive Radio
				      \item 5G New Radio (NR) Cellular
			      \end{itemize}
			\item {\bf Non-contiguous carriers intermodulate}
			      \begin{itemize}
				      \item Caused by nonlinearities in power amplifiers (PAs)
				      \item Undesired spurious emissions (spurs)
				      \item Could interfere with nearby channels
				      \item Self-interference to own receiver when using FDD
				            %frequency-division duplexing
			      \end{itemize}
			%\item{\bf Current 4G chipsets support up to 4 carriers}
			%      \begin{itemize}
			%	      \item Snapdragon 835
			%	      \item 4x20 MHz carrier aggregation downlink, 2x20 MHz uplink
			%      \end{itemize}
			%     \vspace{1mm}
			\item{\textbf{\textit{ Need efficient way to linearize for this scenario}}}
			\item{\textbf{\textit{ DPD requires extra hardware}}}
			      \begin{itemize}
				      \item Extra RX chains
				      \item Larger area
				      \item More Power
			      \end{itemize}
		\end{itemize}  }
	%\block{Power Spectral Density}{
		%\begin{tikzfigure}[\large Intermodulations when broadcasting more than 2 carriers.]
		%\label{fig:fig1}
		%\includegraphics[width=0.95\linewidth]{SubBandIM3s2}
		%\end{tikzfigure}
	%}
	%\block{Related Works}{
	%\Large
	%    \begin{itemize}
	%        \item {\bf Reduce Power}
	%            \begin{itemize}
	%                \item Operate in a more linear PA region
	%                \item Less range and less power efficient
	%            \end{itemize} 
	%                        \vspace{1mm}
	%        \item {\bf Full-Band Digital Predistortion (DPD)}
	%            \begin{itemize}
	%                \item Computationally expensive
	%                \item Does not scale for noncontiguous carriers
	%                \item Requires large sampling rate as carrier spacing grows
	%            \end{itemize}  
	%                        \vspace{1mm}
	%        \item {\bf Sub-Band DPD}
	%            \begin{itemize}
	%                \item Previously explored by the authors with the WARP SDR RF Board
	%                \item Observes and applies DPD to individual spurs
	%                \item Can reduce the necessary sampling rate and complexity
	%                \item Has only been considered for 2 carriers           
	%            \end{itemize}                 
	%    \end{itemize}
	%}
	\block{Main Idea}{\Large
		\begin{itemize}
			\item {\bf Use a lower precision ADC to reduce the area and cost for applying DPD on a UE device}
			      \begin{itemize}
				      \item Iteratively learn coefficients as necessary using adaptive, LMS algorithm.
				      \item Apply them as in Equation 5 to reduce spurious emissions.
			      \end{itemize}
		\end{itemize}
	}

	\block{M\hsct{atlab} Simulation}{\Large
		\begin{itemize}
			\item {\bf LTE-Advanced CA Scenario}
			      \begin{itemize}
				      \item Two, 5 MHz component carriers
				      \item 9th order, parallel Hammerstein PA model
				      \item Fixed point toolbox to emulate ADC
			      \end{itemize}
		\end{itemize}
	}
	\column{0.33}
	\block{Full-band DPD Simulations}{\Large
		\begin{itemize}
			\item {\bf Simulation Architecture}
		\end{itemize}

		\begin{tikzfigure}[]
			\label{fig:fig1}
			\centering
			\includegraphics[width=12in]{../TEX/FullBandIndirect.pdf}
		\end{tikzfigure}

		\begin{itemize}
			\item[] {} %Empty item to allow the 
			      \begin{itemize}
				      \item {\textsc{Matlab} simulation}
				      \item {9th order Parallel Hammerstein PA model}
			      \end{itemize}
		\end{itemize}

		\begin{tikzfigure}[]
			\label{fig:fig1}
			\centering
			\includegraphics[width=12in]{../TEX/FullBandPSD.pdf}
		\end{tikzfigure}

	}



	%\begin{tikzfigure}[]
	%\label{fig:fig1}
	%\centering
	%\includegraphics[]{filename2}
	%\end{tikzfigure}

	%\begin{tikzfigure}[]
	%\label{fig:fig1}
	%\includegraphics[]{filename}
	%\end{tikzfigure}


	\column{0.33}
	\block{Sub-band DPD Simulations}{\Large
		\begin{tikzfigure}[]
			\label{fig:fig1}
			\centering
			\includegraphics[width=12in]{../TEX/SubBand.pdf}
		\end{tikzfigure}
		\begin{tikzfigure}[]
			\label{fig:fig1}
			\centering
			\includegraphics[width=12in]{../TEX/SubBandPSD.pdf}
		\end{tikzfigure}
	}

	\block{G\hsct{nu}Radio Simulator}{\Large
		%\begin{itemize}
		%\item {\bf GNURadio}
		%\begin{itemize}
		%\item Software defined radio (SDR) development platform
		%\item Can work with SDR boards such as the USRP
		%\item Efficient, parallel, real-time CPU implementation 
		%\item Uses python and C++ for easy development and performance 
		%\end{itemize}
		%\vspace{1mm}
		%\item{\bf DPD Simulator}
		%\begin{itemize}
		%\item Custom GNURadio flowgraph, modules, and blocks for performing DPD
		%\item Movable component carriers
		%\item Changeable PA model
		%\item Can add and remove DPD processing
		%\end{itemize}
		%    \begin{tikzfigure}
		%\includegraphics[width=0.98\linewidth]{gnu.png}
		%\end{tikzfigure}

		%\end{itemize}
	}
	%\note[targetoffsetx=0.5in,targetoffsety=-5.5in,width = 6in,connection,rotate=2]{\bf Can change carrier magnitude and placment, PA model, and DPD application dynamically.}



	\block{Future Work}{\Large
		\begin{itemize}
			\item {\bf Main carrier linearization}
			      %\begin{itemize}
			      %    \item Use a variation of the sub-band DPD method to reduce in-band emission. 
			      %\end{itemize}
			\item {\bf Hardware testing with a real PA using the \textsc{Warp} SDR platform}
			      %\begin{itemize}
			      %    \item USRP SDR RF Board
			      %    \item Off-the-shelf, external PAs
			      %\end{itemize}
		\end{itemize}
	}

\end{columns}
\end{document}